\section{Psalm 56 Comments}

\subsection{Numeric Nuggets}
Verse 12 has 13 words, while verses 6 and 12 have 13 unique words. The word ``I'' is found 13 times in Psalm 56.

%\subsection{Psalm 56:1}
%The lengthy title (“Jonathelemrechokim”) signifies “the dove of the distant Terebinths,” according to Bullinger. It is a doctrinal reference to Jews in the Tribulation who have fled to the wilderness where David desired to flee on the “wings of a dove” (see Ps. 55:6). The first four verses are self-explanatory and are largely devotional. However, the enemy who is about to “swallow” him (vs. 1) is not to be ignored, as we have already learned from Isaiah 6:13; Revelation 6:9; and Psalm 30:14. “Many that fight against me” (vs. 2) are the “nations” of Zechariah 14:1–2, so David again is speaking for Israel. “O thou most High” (vs. 2) is instructive, for this is the title used of God in Daniel chapter 7 (six times) by Jewish saints taking over the Kingdom at the Advent. It is found in Daniel chapters 3, 4, and 5 eight times. It is peculiarly a Gentile designation for God as when a Gentile priest in Genesis 14:18 dealt with the first Hebrew (Gen. 14:22). In Deuteronomy 32:8 it is the divine name for One who set up twelve Gentile nations numbered after the children of Jacob. Second Samuel 22:14 fixes the title at the Advent again. But there are other great and precious truths in the passage. Spurgeon’s Treasury has scores of them: “There is fear without trust, trust without fear, and fear with trust”; “manifold mercies, tender mercies, and covenant mercies”; “the language of faith, gratitude and hope,” etc. Devotionally, we may say that our “fighting” (vs. 1) has to be “daily”; the word occurs twice (vss. 1 and 2). Although there are many children of God who, individually, do not have a multitude of human enemies actually “out to get them,” the entire world system -- plus the principalities and powers, plus the ``spiritual wickedness in high places-- (Eph. 6:12) -- are lined up to oppose God, the Bible, Christ, and the child of God. It may not always break out in the raw—like it did between A.D. 500--1700 in Europe, or between A.D.1920--1990 in Russia, or between A.D. 1945--1980 in China -- but the potential is always there. Rome now does only what she can get away with because ``bad press'' has to be avoided in the twentieth century--so far.\cite{Ruckman1992Psalms}

%\subsection{Psalm 56:3}
%''What time I am afraid” (vs. 3) will be the times of persecution, inflation, sickness, war, physical threats, slander, and exposure. “I will trust in thee,” not my job, experience, education, reputation, flesh, religion, weapons, intelligence, or friends. Note that David is somewhat of a “Bibliolator” (as Paul; see Rom. 9:17 and Gal. 3:8), for he not only praises God, but praises what God had recorded in the Scripture (vs. 4), which he exalts above God’s name (Ps. 138:2). America made David’s profession (vs. 4), but then trusted in the coins the profession was printed on; they quickly degenerated from gold to silver (see Dan. 2) and from silver to copper.\cite{Ruckman1992Psalms}

%\subsection{Psalm 56:5}
%See how they did this with Christ’s words in Luke 11:54; Mark 12:13; and Luke 23:2. Peter says that apostates like Hutson, Hudson, Hymers, Horton, Hobbs, Hindson, et al., “wrest...the scriptures” (2 Pet. 3:16). They alter any text they can’t believe or don’t like so that it will say what they want it to say. See for example the RV, RSV, NRSV, NIV, ASV, NASV, NKJV, or any other of the hundred products of the Alexandrian Cult. All atheists, Communists, Catholics, and Fundamental scholars “wrest” God’s words.\cite{Ruckman1992Psalms}

%\subsection{Psalm 56:6}
%Verse 6 is true of most religious conventions, fellowships, councils, associations, and “Congresses.” They make false professions of faith (i.e., “We believe the Bible is the Word of God”), censor material in their schools (out with Burgon, Fuller, Hills, Pickering, and Ruckman), and all come to one unified ecumenical agreement on final and absolute authority (i.e., all truth is relative and subject to the opinions and preferences of “scholars”). “They mark my steps.” You bet they do. There is not one Bible-correcting, apostate Fundamentalist in America who does not keep his eye on every Bible-believing, soulwinning preacher who comes to his attention, and they study the ads in the Sword of the Lord or the Baptist Bible Tribune to prosylete every young Baptist called to preach in order to destroy him with Nicolaitanism: ``the absolute authority of the school'' over the local church and the Book.\cite{Ruckman1992Psalms}

%\subsection{Psalm 56:10}
%Verses 9 and 10 are self explanatory. Note again, David praises the written words that God had spoken and recorded (2 Sam. 23:2). Note also, the peculiar notation that David is the “sweet psalmist of Israel” (2 Sam. 23:1). You cannot disconnect much of David’s Psalm from the nation of Israel. “God is for me” (compare Rom. 8:31–32 in the New Testament). Note, “in God” and “in the Lord” without being “in Christ” (see Acts 17:28). No Old Testament saint was “in Christ,” despite the fanatical ravings of people like Curtis Hutson, John R. Rice, and the faculties and staffs of BBC, BJU, TTS, PCS, and other Alexandrian offshoots. Verse 11 should be claimed as a promise. Christ enforces this with Matthew 10:28, which see. \cite{Ruckman1992Psalms}

%\subsection{Psalm 56:12}
%``Thy vows are upon me” (vs. 12) in the sense of Jonah 2:9. He will fulfill what he vowed (see Eccl. 5:4–6). Although the historical content of verse 13 is plainly David staying alive physically, there is beautiful application to the New Testament child of God who has “eternal security.” Our salvation enables us to walk with God “in the light” (see 1 John 1:7) and “the living” are those who have life “more abundantly” (John 10:10). Our soul’s deliverance (see vs. 13) has been from the “second death,” (Rev. 2:11) which Christ calls the destruction of the soul (Matt. 10:28), and with it came the deliverance of our feet (Heb. 12:13) which can now walk in the “paths of righteousness” (Ps. 23:3) instead of going back into sin. The question of Psalm 56:13 is answered in Psalm 116:8, which see. \cite{Ruckman1992Psalms}

