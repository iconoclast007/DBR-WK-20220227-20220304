\chapter{Psalm 53}


\marginpar{\scriptsize \centering \fcolorbox{black}{yellow}{\textbf{FOOLS AND SAINTS}}\\ (Psalm 53:1-6) \begin{compactenum}[I.][8]
    \item A \textbf{Foolish Saying} \index[scripture]{Psalms!Psa 053:01} (Psalm 53:1)
    \item A \textbf{Frustrating Search} \index[scripture]{Psalms!Psa 053:02} (Psalm 53:2)
    \item \textbf{Filthy Sinners}  \index[scripture]{Psalms!Psa 053:03} (Psalm 53:3)
    \item \textbf{Fearful and Scattered}  \index[scripture]{Psalms!Psa 053:05} (Psalm 53:5)
    \item \textbf{Future Society}  \index[scripture]{Psalms!Psa 053:06} (Psalm 53:6)
    \item \textbf{Family of Saints}  \index[scripture]{Psalms!Psa 014:05} (Psalm 14:5)
\end{compactenum}}




\footnote{\textcolor[cmyk]{0.99998,1,0,0}{\hyperlink{TOC}{Return to end of Table of Contents.}}}\textcolor[cmyk]{0.99998,1,0,0}{The fool hath said in his heart, \emph{There} \emph{is} no God. Corrupt are they, and have done abominable iniquity: \emph{there} \emph{is} none that doeth good.}\footnote{Psalm 53 delineates the conditions on earth through the Tribualtion. \begin{compactenum}
\item Verse 1 -- shows a plethora of professing atheists at the beginning of the Tribulation, before the Antichrist declares that he is God in the flesh.
\item  Verse 2 -- In contrast, a man of understanding will seek God by prayer and the Book, by watching, and by studying God’s ways and works.
\item Verse 3 -- shows what is going on in the last half of Daniel’s Seventieth Week. It cannot be a reference to Noah’s day, Abraham’s day, or the Church Age; it will have to be in the Tribulation.
\item Verse 4 -- Would a man with knowledge fail to call on God? Could he eat the literal flesh and blood of a human being?
\item Verse 5 -- shows the populace fearing idols and gods when there is no reason to fear them (2 Kings 17:7; Exod. 23:24), and then in the middle of the verse we find God’s judgment on the Antichrist’s troops who are entangled with those who ``do exploits'' (Dan. 11:28) during the Tribulation.
\item Verse 6 -- The victories of verse 5 are not complete, so the remnant is still praying for the Advent in verse 6. When salvation comes in the Tribulation, it comes ``out of Zion'' (see Psalm 14:7, 50:2; Joel 2:1, 32, and 3:16), not Calvary or Golgotha.
\end{compactenum}}\footnote{An atheist's problem is in his heart and not in his head  (see Psalm 10:6, 11, 13). Any fool who has any bit of rationality will know that there has to be a God somehwere. The issue is when one does not want God to be there to interfere with what you are doing. \cite{Ruckman1992Psalms}} \footnote{\textbf{Proverbs 18:1--3} -- Through desire a man, having separated himself, seeketh and intermeddleth with all wisdom.  [2]  A fool hath no delight in understanding, but that his heart may discover itself. [3] When the wicked cometh, then cometh also contempt, and with ignominy reproach. \cite{Ruckman1992Psalms} }
[2] \textcolor[cmyk]{0.99998,1,0,0}{God looked down from heaven upon the children of men, to see if there were \emph{any} that did understand, that did seek God.}\footnote{[RUCKMAN] In verse 2 we learn that a ``man of understanding'' will seek God by prayer, the Book, and watching and studying God’s works day and night. The terrorized people of verse 5 are terrorized because they had not called on God (vs. 4). What a man loves and what he fears are the measure of the man -- any man. No man on earth is what he professes, tolerates, stands for, makes, earns, deserves, is credited with, says, or writes. Every man on earth is what he loves and what he fears. Wisdom is simply knowing when to be afraid and when not to be afraid. ``The fear of the Lord is the beginning of wisdom'' (Prov. 9:10); ``the fear of man bringeth a snare'' (Prov. 29:25). Bob Jones III’s profession in having a ``godly institution'' that is a ``Bastion of Orthodoxy'' for ``Congresses'' of militant Fundamentalists has nothing to do with anything. If a man fears ridicule from intellectuals and loves the ``preeminent place'' as a ``defender of the faith'' he is an apostate coward. You are what you love and what you fear, and so are the other five billion inhabitants of “planet earth.” Pilate feared the crowd (Matt. 27:24), Ahab feared Elijah (1 Kings 21:27), Herod feared ridicule (Matt. 14:9), and Israel feared gods that were no gods (2 Kings 17:7). Judas loved money (John 12:6), as did Gehazi (2 Kings 5:20); Ahab loved property (1 Kings 21:6); Ananias and Saphirra loved a reputation for being spiritual when they weren't (Acts 5:1--4); and the Man of Sin will love POWER and WORSHIP (Dan. 11:36--38). You ARE what you love and what you fear. When a Christian educator loves property and buildings more than the souls of men, he is a carnal, worldly FOOL (Prov. 30:5, 11:30). When a Bible teacher fears the criticism of his peers and loves his own reputation as a ``great scholar,'' he is a backslidden reprobate. You are what you love and what you fear.\cite{Ruckman1992Psalms} }
[3] \textcolor[cmyk]{0.99998,1,0,0}{Every one of them is gone back: they are altogether become filthy; \emph{there} \emph{is} none that doeth good, no, not one.}
[4] \textcolor[cmyk]{0.99998,1,0,0}{Have the workers of iniquity no knowledge? who eat up my people \emph{as} they eat bread: they have not called upon God.}
[5] \textcolor[cmyk]{0.99998,1,0,0}{There were they in great fear, \emph{where} no fear was: for God hath scattered the bones of him that encampeth \emph{against} thee: thou hast put \emph{them} to shame, because God hath despised them.}\footnote{If a man fears ridicule from intellectuals and loves the “preeminent place” as a “defender of the faith” he is an apostate coward. You are what you love and what you fear, and so are the other five billion inhabitants of “planet earth.” Pilate feared the crowd (Matt. 27:24), Ahab feared Elijah (1 Kings 21:27), Herod feared ridicule (Matt. 14:9), and Israel feared gods that were no gods (2 Kings 17:7). Judas loved money (John 12:6), as did Gehazi (2 Kings 5:20); Ahab loved property (1 Kings 21:6); Ananias and Saphirra loved a reputation for being spiritual when they weren’t (Acts 5:1–4); and the Man of Sin will love POWER and WORSHIP (Dan. 11:36–38). What a man loves and what he fears are the measure of the man—any man. No man on earth is what he professes, tolerates, stands for, makes, earns, deserves, is credited with, says, or writes. Every man on earth is what he loves and what he fears. Wisdom is simply knowing when to be afraid and when not to be afraid. “The fear of the Lord is the beginning of wisdom” (Prov. 9:10); “the fear of man bringeth a snare” (Prov. 29:25). \cite{Ruckman1992Psalms}}
[6] \textcolor[cmyk]{0.99998,1,0,0}{Oh that the salvation of Israel \emph{were} \emph{come} out of Zion! When God bringeth back the captivity of his people, Jacob shall rejoice, \emph{and} Israel shall be glad.}


