\section{Psalm 57 Outlines}

\subsection{My Outlines}

\subsubsection{Everyday Attributes of God that I Like}
\textbf{Introduction:} Psalm 57 is one of those Psalms that speak of dark days for David. It speaks of a time when David was in hiding from Saul in a cave, perhaps the cave of Adullam spoken in 1 Samuel 22, or 2 Samuel 23 or 1 Chronicles 11.  Or perhaps a cave in En-gedi spoken of in 1 Samuel 24. In anyc ase, David is running for his life and in hiding from one who wishes to destroy him.  In fact, Psalm 57 pictures the time of Tribulation, and verse 23 even shows a great resuce of the Jews in the middle of the Trobulation. John Phillips speaks of David's \textbf{Calamites} in the psalm, and his \textbf{Crises}, but he also speaks on David's \textbf{Confidence}.\cite{Phillips2001PsalmsV1} I've had dark times in life, although never to the extent of having to hide from someone who is seeking to kill me, and I'm sure we all have. I am getting older and am starting to experience some health issues -- I definitely do not have the strength and stamina that I used to. I seem to need more naps these days. But one thing that is not failing for me is my relationship with God. And this is mostly because of who God is.  As I spend more time as a Christina, I appreciate God even more, as I learn about who God really is. During my 39 years as a Christian, I come to know some of the ``Everyday attributes of God.'' In Bible school, studying theology, we learned that God is Omnipresent, He is Omniscient, and he is Ominpotent.  He is also holy. But for a few minutes, here, I want to cover some of these everyday attributes. I'll start by saying that this is NOT a complete list. I won't do God justice in my descriptions. And, I haven't worked out the theology for them.
\index[speaker]{Keith Anthony!Psalm 057 (Everyday Attributes of God that I Like}
\index[series]{Psalms (Keith Anthony)!Psalm 057 (Everyday Attributes of God that I Like}
\index[date]{2017/02/03!Psalm 057 (Everyday Attributes of God that I Like) (Keith Anthony)}
\begin{compactenum}[I.][19]
    \item \textbf{Abundantness} \index[scripture]{Philippians!Phil 04:19} (Philippians 4:19) But my God shall supply all your need according to his riches in glory by Christ Jesus. God is the pantry that is never empty. He is the well that never runs dry. He is the oil barrel of the widow of Zarepath.
    \item \textbf{Aboveness} I am happy that I have a God who is has aboveness.  He is above the fry. the petty squabbles and even the big ones, do not affect God.  They don't derail his plan. they don't sidetrack his plan. the don't defile his plan.
    \item \textbf{Alongsideness} \index[scripture]{SoS!SoS 04:08}\index[scripture]{Psalms!Psa 057:11} (Psalm 57:11) speaks of, in a romatic sense,the joy of having that special one with you.  ``Come with me from Lebanon, my spouse, with me from Lebanon: look from the top of Amana, from the top of Shenir and Hermon, from the lions’ dens, from the mountains of the leopards. [9] Thou hast ravished my heart, my sister, my spouse; thou hast ravished my heart with one of thine eyes, with one chain of thy neck.'' When everyone you care about is gone, for a Christian God will be right there beside you.
    \item \textbf{Aforeness} During the travels of the Israelites to the Promised Land, they were lead by the Lord%\index[scripture]{Psalms!Psalm 057:05}\index[scripture]{Psalms!Psalm 057:11} (Psalm 57:5, 11)
    \item \textbf{Aroundness} God has ``aroundness''. It is that difficult to see the existence of, evidence of, and presence of God if you are looking.%\index[scripture]{Psalms!Psalm 057:05}\index[scripture]{Psalms!Psalm 057:11} (Psalm 57:5, 11)
    \item \textbf{Absoluteness} %\index[scripture]{Psalms!Psalm 057:05}\index[scripture]{Psalms!Psalm 057:11} (Psalm 57:5, 11)
    \item \textbf{Amongness} \index[scripture]{Matthew!Matt 18:20}(Matt 18:20) Tells us that wherever two or three believers are gathered together is His name, go is among them.   \index[scripture]{Revelation!Rev 01:03} Revelation 1:3 tells is that Jesus is in the midst of the churches.  It is wonderful to kow that the living God, the living Saviour meets with Christians.
    \item \textbf{Awareness} God's omniscience means that he is completely aware of everything about me, everything around me, everything affecting me, everything in store for me, every decision I will have to make (and the results of them)%\index[scripture]{Psalms!Psalm 057:05}\index[scripture]{Psalms!Psalm 057:11} (Psalm 57:5, 11)
    \item \textbf{Alwaysness} This attribute I guess is God's immutability restated. God is completely reliable.God is the guarantee. 100 \% certain. %\index[scripture]{Genesis!Genesis 06:03}\index[scripture]{Exodus!Exodus 27:20}\index[scripture]{2 Thessalonians!2 Thessalonians 3:16} (Genesis 6:3, Exodus 27:20, 2 Thessalonians 3:16)
    \item \textbf{Attentivness} I think of a brand new baby and the attention a mother gives it. Every need is anticipated and met. Everything thing is done with tenderness and 100 per cent focus. God is that way, that he can give every one of his children complete focus, all the time.
    \item \textbf{Availability} \index[scripture]{Jeremiah!Jer 33:03} (Jeremiah 33:3) says ``Call unto me, and I will answer thee, and shew thee great and mighty things, which thou knowest not.''  To the soul who wants God, he is always available. A christian knows that  from scripture. He knows that from experience. He knows that from the testimony of other saints. If you do not HAVE God, He is available right now. 
\end{compactenum}

\subsection{Outlines from Others}

\subsubsection{Hide Me, Oh My Saviour, Hide}
%\textbf{Introduction:} Psalm 57 is one of those Psalms that speak of dark days for David. It speaks of a time when David was in hiding from Saul in a cave, perhaps the cave of Adullam spoken in 1 Samuel 22, or 2 Samuel 23 or 1 Chronicles 11.  Or perhaps a cave in En-gedi spoken of in 1 Samuel 24. In anyc ase, David is running for his life and in hiding from one who wishes to destroy him.  In fact, Psalm 57 pictures the time of Tribulation, and verse 23 even shows a great resuce of the Jews in the middle of the Trobulation. John Phillips speaks of David's \textbf{Calamites} in the psalm, and his \textbf{Crises}, but he also speaks on David's \textbf{Confidence}.\cite{Phillips2001PsalmsV1} I've had dark times in life, although never to the extent of having to hide from someone who is seeking to kill me, and I'm sure we all have. I am getting older and am starting to experience some health issues -- I definitely do not have the strength and stamina that I used to. I seem to need more naps these days. But one thing that is not failing for me is my relationship with God. And this is mostly because of who God is.  As I spend more time as a Christina, I appreciate God even more, as I learn about who God really is. During my 39 years as a Christian, I come to know some of the ``Everyday attributes of God.'' In Bible school, studying theology, we learned that God is Omnipresent, He is Omniscient, and he is Ominpotent.  He is also holy. But for a few minutes, here, I want to cover some of these everyday attributes. I'll start by saying that this is NOT a complete list. I won't do God justice in my descriptions.
\textbf{Source: }John Phillips
\index[speaker]{John Phillips!Psalm 057 (Hide Me, Oh My Saviour, Hide}
\index[speaker]{Psalms (John Phillips)!Psalm 057 (Hide Me, Oh My Saviour, Hide}
\index[date]{2017/02/03!Psalm 057 (Hide Me, Oh My Saviour, Hide\footnote{John Phillips}) (John Phillips)}
\begin{compactenum}[I.][3]
    \item The \textbf{Calamities which Thronged Him} \index[scripture]{Psalms!Psa 057:01--03} (Psalm 57:1--3)
	\begin{compactenum}[A.][3]
		\item Lord, Hide Me \index[scripture]{Psalms!Psa 057:01} (Psalm 57:1)
		\item Lord, Hear Me \index[scripture]{Psalms!Psa 057:02} (Psalm 57:2)
		\item Lord, Help Me \index[scripture]{Psalms!Psa 057:03} (Psalm 57:3)
	\end{compactenum}
    \item The \textbf{Crises which Threatened Him} \index[scripture]{Psalms!Psa 057:04--06} (Psalm 57:4--6)
	\begin{compactenum}[A.][3]
		\item The Seriousness of his Situation \index[scripture]{Psalms!Psa 057:04} (Psalm 57:4)
		\item The Sovereignty of his Saviour \index[scripture]{Psalms!Psa 057:05} (Psalm 57:5)
		\item The Significance of his Salvation \index[scripture]{Psalms!Psa 057:06} (Psalm 57:6)
	\end{compactenum}
    \item The \textbf{Confidence which Thrillled Him} \index[scripture]{Psalms!Psa 057:07--11} (Psalm 57:7--11)
	\begin{compactenum}[A.][3]
		\item A Willing Confidence \index[scripture]{Psalms!Psa 057:07} (Psalm 57:7)
		\item A Witnessing Confidence \index[scripture]{Psalms!Psa 057:08--09} (Psalm 57:8--9)
		\item A Worshipping Confidence \index[scripture]{Psalms!Psa 057:10--11} (Psalm 57:10--11)
	\end{compactenum}
\end{compactenum}