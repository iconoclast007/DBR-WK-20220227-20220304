\subsection{Outlines from Others}


\subsubsection{Hide Me, Oh My Saviour, Hide}
%\textbf{Introduction:} Psalm 57 is one of those Psalms that speak of dark days for David. It speaks of a time when David was in hiding from Saul in a cave, perhaps the cave of Adullam spoken in 1 Samuel 22, or 2 Samuel 23 or 1 Chronicles 11.  Or perhaps a cave in En-gedi spoken of in 1 Samuel 24. In anyc ase, David is running for his life and in hiding from one who wishes to destroy him.  In fact, Psalm 57 pictures the time of Tribulation, and verse 23 even shows a great resuce of the Jews in the middle of the Trobulation. John Phillips speaks of David's \textbf{Calamites} in the psalm, and his \textbf{Crises}, but he also speaks on David's \textbf{Confidence}.\cite{Phillips2001PsalmsV1} I've had dark times in life, although never to the extent of having to hide from someone who is seeking to kill me, and I'm sure we all have. I am getting older and am starting to experience some health issues -- I definitely do not have the strength and stamina that I used to. I seem to need more naps these days. But one thing that is not failing for me is my relationship with God. And this is mostly because of who God is.  As I spend more time as a Christina, I appreciate God even more, as I learn about who God really is. During my 39 years as a Christian, I come to know some of the ``Everyday attributes of God.'' In Bible school, studying theology, we learned that God is Omnipresent, He is Omniscient, and he is Ominpotent.  He is also holy. But for a few minutes, here, I want to cover some of these everyday attributes. I'll start by saying that this is NOT a complete list. I won't do God justice in my descriptions.

\index[speaker]{John Phillips!Psalm 057 (Hide Me, Oh My Saviour, Hide}
\index[speaker]{Psalms (John Phillips)!Psalm 057 (Hide Me, Oh My Saviour, Hide}
\index[date]{2017/02/03!Psalm 057 (Hide Me, Oh My Saviour, Hide\footnote{John Phillips}) (John Phillips)}
\begin{compactenum}[I.][3]
    \item The \textbf{Calamities which Thronged Him} \index[scripture]{Psalms!Psa 057:01--03} (Psalm 57:1--3)\footnote{John Phillips, Exploring the Psalms, Vol I.\cite{Phillips2001ExploringPsalms1}}
	\begin{compactenum}[A.][3]
		\item Lord, Hide Me \index[scripture]{Psalms!Psa 057:01} (Psa 57:1)
		\item Lord, Hear Me \index[scripture]{Psalms!Psa 057:02} (Psa 57:2)
		\item Lord, Help Me \index[scripture]{Psalms!Psa 057:03} (Psa 57:3)
	\end{compactenum}
    \item The \textbf{Crises which Threatened Him} \index[scripture]{Psalms!Psa 057:04--06} (Psa 57:4--6)
	\begin{compactenum}[A.][3]
		\item The Seriousness of his Situation \index[scripture]{Psalms!Psa 057:04} (Psa 57:4)
		\item The Sovereignty of his Saviour \index[scripture]{Psalms!Psa 057:05} (Psa 57:5)
		\item The Significance of his Salvation \index[scripture]{Psalms!Psa 057:06} (Psa 57:6)
	\end{compactenum}
    \item The \textbf{Confidence which Thrillled Him} \index[scripture]{Psalms!Psa 057:07--11} (Psa 57:7--11)
	\begin{compactenum}[A.][3]
		\item A Willing Confidence \index[scripture]{Psalms!Psa 057:07} (Psa 57:7)
		\item A Witnessing Confidence \index[scripture]{Psalms!Psa 057:08--09} (Psa 57:8--9)
		\item A Worshipping Confidence \index[scripture]{Psalms!Psa 057:10--11} (Psa 57:10--11)
	\end{compactenum}
\end{compactenum}